\section{Detection \& Eviction}
\label{sec:detection}

Based on the proposed internal attack along with the evaluation of the attack's impact, some basic ideas for the detection and eviction scheme can be highlighted.
In this section, the aim is to discuss the essential requirements that the detection and eviction mechanisms should adopt.

\subsection{Detection Mechanism}
Given the source's upload bandwidth, it is feasible by even low budget attacker to fully occupy its headnodes. 
Hence, cutting out the source can be achieved.
Accordingly, the detection mechanism should:
\begin{enumerate}
 \item Include a feedback scheme so that peers, especially the source, can be informed about other peers' satisfaction level of the stream. 
 Most importantly, the challenger peer $u$ must be able to have a wider view of the network, i.e., be informed about other peers' satisfaction level in the mesh.
 
 \item Each peer should maintain a satisfaction list to suspect peers not delivering chunks available in their buffer maps. 
 This way, even malicious peers that do NOT consistently refuse to deliver the right chunks to other peers can be eventually suspected according to a specific system threshold parameter $\alpha$.
 
 \item Allow the source to collect the feedback from peers that exist behind its surrounding headnodes, i.e., peers with $hops>1$ away from the source.
 
 \item Develop/use a signing mechanism for the feedback messages so that malicious peers can not manipulate these messages, or get suspected otherwise.
 
 \item Suspect malicious peers whenever they: (i) alter the messages,(ii) just refuse to deliver it to $u$ or (iii) they acquire a satisfaction level below $\alpha$.
  
 
\end{enumerate}

\subsection{Eviction Mechanism}
Once the detection mechanism announces a peer as malicious, the eviction mechanism should adopt the following strategy:

\begin{enumerate}
 \item Allows the source to reach to those $hops>1$ peers once the source detects that it is surrounded with malicious headnodes, i.e., rapid changing of the source's headnodes.
 \item Forces each peer to collect evidence (the responder $v$ replies) about the event of declaring another as malicious and thus, trigger a counter detection in case a malicious peer wants to evict a benign one.
 \item Maliciously claimed peers should be added to a global blacklist, or even locally dropped from the challenger's neighbor list?
\end{enumerate}

