\begin{abstract}

The data dissemination model used in Pull-based systems utilizes the performance and enriches the scalability of P2P streaming overlay's size given the tight delay in-tolerance constraints.
Nevertheless, a lot of research has been conducted that highlights the susceptibility of such systems to a diverse set of attacks due to the absence/inapplicability of a central/trusted authority to track the content of both, the buffer-maps and the actual exchanged video chunks.

In this work, we address the threat of attacking the closest nodes to the source and thus, causing a severe discontinuity of the stream.
In such attack, we show in our simulation studies that a very small budget is sufficient to fully intercept the video chunks before the dissemination process between peers starts.

We propose a detection mechanism that, with high accuracy, isolates the detected malicious peers and thus, allows for a speedy recovery of the stream.
Through both a theoretical and a simulation-based analysis, we validate that the detection mechanism is able to correctly detect malicious peers with up to x\% while inducing a small overhead of y\%.

\end{abstract}