\begin{abstract}

Pull-based data dissemination technique is the most commonly used architecture in online video streaming P2P systems.
Although a lot of research had been conducted that focus on mitigating various forms of attacks, the class of inference attacks: the attacker is capable of inferring the peers directly connected to the source,
still constitutes a remarkable threat on streaming P2P systems.

We focus in this work on internal inference attacks, where the attacker first infer the overlay's topology before conducting a buffer-map $BM$ cheating attack using malicious peers located at the closest positions to the source.
Through the simulations, we demonstrate the feasibility of conducting such attack with few malicious peers and how drastically the P2P overlay is impacted.

Accordingly, we propose a two-fold detection mechanism that is capable of restoring the benign state of the overlay and detect malicious peers conducting various $BM$ cheating behaviors in the worst case scenario where 100\% of the sources neighbors are malicious.
The proposed mechanism allows peers to collaboratively decide and thus, file a complain to the source who in turn replaces nearby suspicious peers.

% Simultaneously, peers locally sanitize their contact lists from peers who conduct adversarial $BM$ cheating behaviors.

Through both a theoretical and a simulation-based analysis, we validate that the detection mechanism is able to correctly detect malicious peers with up to 90\% while inducing a small overhead of 8\%.

\end{abstract}