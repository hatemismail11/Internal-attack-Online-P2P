\begin{abstract}

% Pull-based data dissemination technique is the most commonly used architecture in online video streaming P2P systems.
% Although a lot of research had been conducted that focus on mitigating various forms of attacks, the class of inference attacks: the attacker is capable of inferring the peers directly connected to the source,
% still constitutes a remarkable threat on streaming P2P systems.
% 
% We focus in this work on internal inference attacks, where the attacker first infer the overlay's topology before conducting a buffer-map $BM$ cheating attack using malicious peers located at the closest positions to the source.
% Through the simulations, we demonstrate the feasibility of conducting such attack with few malicious peers and how drastically the P2P overlay is impacted.
% 
% Accordingly, we propose a two-fold detection mechanism that is capable of restoring the benign state of the overlay and detect malicious peers conducting various $BM$ cheating behaviors in the worst case scenario where 100\% of the sources neighbors are malicious.
% The proposed mechanism allows peers to collaboratively decide and thus, file a complain to the source who in turn replaces nearby suspicious peers.
% 
% Through both a theoretical and a simulation-based analysis, we validate that the detection mechanism is able to correctly detect malicious peers with up to 80-90\% while inducing a small overhead of ~8\%.

% Simultaneously, peers locally sanitize their contact lists from peers who conduct adversarial $BM$ cheating behaviors.

Pull-based data dissemination is one of the most used communication approaches in online video streaming P2P systems, and also constitutes a prominent target for security breaches.
While extensive research exists on mitigating varied attack types, the class of inference attacks (where the attacker is capable of inferring the peers directly connected to the source), still constitutes a major open threat for streaming P2P systems.

Our research focuses on the critical class internal inference attacks, where the attacker first infers the overlay's topology before conducting a buffer-map ($BM$) cheating attack using malicious peers located at the positions closest to the source. 
We demonstrate the feasibility of conducting such attacks, and also show how very few malicious peers can drastically compromise the P2P overlay.

Accordingly, we propose a two-fold detection mechanism that is capable of (a) detecting the malicious peers conducting various $BM$ cheating behaviors in the worst case scenario where 100\% of the sources neighbors are malicious, and (b) restoring the the overlay to a benign state. 
The proposed mechanism allows peers to collaboratively detect and notify the source, which in turn replaces the suspected peers.

% Simultaneously, peers locally sanitize their contact lists from peers who conduct adversarial $BM$ cheating behaviors.

Using a combined theoretical and simulation-based analysis, we validate that the detection mechanism is able to detect malicious peers with up to 80-90\% accuracy while inducing a small overhead of ~8\%.

\end{abstract}