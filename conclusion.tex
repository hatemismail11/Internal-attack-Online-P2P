\section{Conclusion \& Future Work}
\label{sec:conclusion}
In this work\footnote{Research supported in part by EC H2020 CIPSEC GA \#700378 and BMBF TUD-CRISP}, we focus on the class of internal inference attacks for pull-based overlays where the attacker conducts a $BM$ cheating attack after placing malicious peers as headnodes.
We show that the attack severity significantly increases the chunk loss ratio, accompanied by low satisfaction level experienced by benign peers.
As a countermeasure, we propose a detection mechanism where peers are able to collaboratively file a complaint to the source when their average aggregated satisfaction drops below a certain threshold so the source can replace suspicious headnodes.
% Accordingly, the source can replace specific headnodes that is most likely to be malicious.

Our simulations show that the detection mechanism is capable of restoring \~95-100\% of peers satisfaction level while removing \~80-90\% of malicious headnodes from its neighbor list while inducing a low overhead of approximately 8\%.
As an ongoing work, we focus on evaluating the resilience of our approach against various $BM$ cheating strategies and integrating anonymous monitoring for proactive defense.
