\section{Internal Attack Model}
\label{sec:Attack}

Here I briefly describe how the internal attack's works.
Assume an attacker with budget $x$ is capable of joining the overlay as early as possible and thus, assigns a fraction $Hx$, where $H\in [0,1]$, of its resources as headnodes.
The remaining malicious peers $MN=x-Hx$ are assigned as neighbors to the $Hx$ malicious headnodes, i.e., those peers are not headnodes, they are connected as neighbors to the malicious headnodes. 

Once a malicious peer $m\in M$ receives a video-chunk from the source or even through another benign peer $b \in B$, $m$ drops the packet. 
Simultaneously, $m$ keeps on sending fake buffer-maps $BM$ to other benign peers claiming that it has those video chunks already. 
This is the current status of the attack. Other adversarial behaviors are also possible to be included, if needed.

For the proposed attack, malicious peers are also characterized by the following:
\begin{enumerate}
 \item Malicious peers maintain another list (other than the normal neighbor list) containing information about all other malicious peers and their status (active, suspected or evicted).
 \item They are aware of the SWAP process. 
 Hence, they fully collude in the sense that they inform each other about what and when to nominate each other to another benign peers or even to the source (if needed).
 \item Concerning SWAP, they are capable of dropping or even manipulating the number of hops in the nomination requests. Accordingly, they can prevent the source from getting benign nominations.
 \item They do NOT churn out, i.e., malicious peers can exploit the source/ benign peers' neighbor list for the whole simulation.
\end{enumerate}



